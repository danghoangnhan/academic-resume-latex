%-------------------------
% Resume in Latex
% Author : Ran Cheng
% Adapted from: Indu dwivedi, Sourabh Bajaj
% License : MIT
%------------------------

\documentclass[letterpaper,10pt]{article}

\usepackage{latexsym}
\usepackage[empty]{fullpage}
\usepackage{titlesec}
\usepackage{marvosym}
\usepackage[usenames,dvipsnames]{color}
\usepackage{verbatim}
\usepackage{enumitem}
\usepackage[pdftex, hidelinks]{hyperref}
\usepackage{fancyhdr}
\usepackage[charter]{mathdesign} % Bitstream Charter
% \usepackage{newpxtext,newpxmath} % Palatino
\usepackage{longtable}
\usepackage{graphicx}
\usepackage{array}
\usepackage{multirow}
\usepackage{xcolor}
\usepackage{bibentry}
\pagestyle{fancy}
\fancyhf{} % clear all header and footer fields
\fancyfoot{}
\renewcommand{\headrulewidth}{0pt}
\renewcommand{\footrulewidth}{0pt}

% bibtex for publication
\bibliographystyle{plain}
\nobibliography{resume.bib}

% Adjust margins
\addtolength{\oddsidemargin}{-0.50in}
\addtolength{\evensidemargin}{-0.50in}
\addtolength{\textwidth}{1in}
\addtolength{\topmargin}{-.5in}
\addtolength{\textheight}{1.0in}

% Define colors
\definecolor{linkblue}{RGB}{111, 153, 222}
\definecolor{titleblue}{RGB}{46, 116, 181}
\urlstyle{same}

\raggedbottom
\raggedright
\setlength{\tabcolsep}{0in}

% Sections formatting
\titleformat{\section}{
  \vspace{-6pt}\scshape\raggedright\large
}{}{0em}{}[\color{black}\titlerule \vspace{-5pt}]

%-------------------------
% Custom commands
\newcommand{\resumeItem}[2]{
  \item\small{
    \textbf{#1}{: #2 \vspace{-2pt}}
  }
}

\newcommand{\resumeItemNoBullet}[2]{
  \item[]\small{
    \hspace{-9pt}\textbf{#1}{: #2 \vspace{-6pt}}
  }
}

\newcommand{\resumeSubheading}[4]{
  \vspace{-1pt}\item[]
  \begin{tabular*}{0.98\textwidth}{l@{\extracolsep{\fill}}r}
      \hspace{-10pt}\textbf{#1} \& #2 \\
      \hspace{-10pt}\textit{\small#3} \& \textit{\small #4} \\
    \end{tabular*}\vspace{-5pt}
}

\newcommand{\resumeSubItem}[2]{\resumeItem{#1}{#2}\vspace{-4pt}}

\renewcommand{\labelitemii}{$\circ$}

\newcommand{\resumeSubHeadingListStart}{\begin{itemize}[leftmargin=*]}
\newcommand{\resumeSubHeadingListEnd}{\end{itemize}}
\newcommand{\resumeItemListStart}{\begin{itemize}}
\newcommand{\resumeItemListEnd}{\end{itemize}\vspace{-5pt}}

% custom commands
\newcommand{\shorterSection}[1]{\vspace{-10pt}\section{#1}}

%-------------------------------------------
%%%%%%  CV STARTS HERE  %%%%%%%%%%%%%%%%%%%%%%%%%%%%


\begin{document}

%----------HEADING-----------------
% you can generate your own qr code here: https://www.the-qrcode-generator.com/
% and convert the svg image you exported to pdf here: https://cloudconvert.com/svg-to-pdf
% then import the graph in the title like this:

\begin{table}[]
\begin{tabular*}{\textwidth}{lc@{\extracolsep{\fill}}r}
\begin{tabular}{l}
\textbf{\huge \textcolor{titleblue}{Daniel Tu}} \\
\\
Ai Engineer, Software Engineer
\end{tabular}  \&  \& \begin{tabular}{@{}rr@{}} \textcolor{titleblue}{\includegraphics[width=0.017\linewidth]{imgs/location.pdf}} Jingde Rd, Xinzhuang District, New Taipei City, Taiwan 242 \& \multirow{3}{*}{\includegraphics[width=0.096\linewidth]{imgs/qrcode.png}} \\
\includegraphics[width=0.017\linewidth]{imgs/email.pdf} danghoangnhan.1@gmail.com                     \&                   \\
\includegraphics[width=0.017\linewidth]{imgs/home.pdf}danghoangnhan.github.io\href{danghoangnhan.github.io}                            \&                   \\
\includegraphics[width=0.017\linewidth]{imgs/phone.pdf} +886 906-276-967                                   \&                   
\end{tabular}  \\ 
\end{tabular*}
\end{table}

\vspace*{-10mm}


%-----------EDUCATION-----------------
\shorterSection{\textcolor{titleblue}{Education}}
  \resumeSubHeadingListStart
    \resumeSubheading
      {Fu Jen Catholic University}{New Taipei city, Taiwan}     {Master of Science in Computer Science}{Sep 2022 - Aug 2024}{
        \resumeItemNoBullet{Thesis}{Enhancing Biomedical Document Analysis with Layout-Aware Multitask Model}
        \resumeItemNoBullet{Field research}{Optical Character Recognition, Document Understanding, Large Language Model, Federated Learning,Object Detection}

    }
%     \resumeSubheading
%       {Coursera}{Online}     {Honored Degrees \& Long-term Community Contributor}{Aug 2015 - Aug 2017}{
%       \resumeItemNoBullet{Completed Courses}{Deep Learning Specialization (
% DeepLearning.AI), TensorFlow: Advanced Techniques Specialization (UPenn), Machine Learning (Stanford)}
%     }
    \resumeSubheading
      {Fu Jen Catholic University}{New Taipei city, Taiwan}     {Bachelor of Engineer, Software Engineering;}{Sep 2018 - Aug 2022}{
%       \resumeItemNoBullet{Honors and Awards}{Outstanding Diploma thesis, National Aspiration Fellowship, Second Class Prize Fellowship, Social Activism Award, IBM
% Outstanding Contribution Award, Microsoft Imagine Cup, FTC (First Tech Challenge, a Robot Competition Conference) Technician}
    \resumeItemNoBullet{Thesis}{Reinforcement Learning Approach for the Restaurant Meal Delivery Problem using Value-inserting Method}
    \resumeItemNoBullet{Field research}{Reinforcement learning}
    }
  \resumeSubHeadingListEnd

%-----------ACADEMIC PROJECTS AND INTERNSHIPS-----------------
\shorterSection{\textcolor{titleblue}{Development Skill}}
  \resumeSubHeadingListStart
    \small{
    \item{\textbf{Programming Language}{: Python, C++, Java, Golang}
    \hfill
    \textbf{Devops}{: Docker, Jenkins, AWS CDK, GCP}}
    \vspace{-8pt}
    \item{\textbf{Backend}{: Django, Spring}
    \hfill
    \textbf{Mobile}{: Android Studio}}
    \vspace{-8pt}
    \item{\textbf{Database}{: Mysql, Mongodb, Redis, Kafka}
    \hfill
    \textbf{Version Control}{: Git}}
    \vspace{-8pt}
    \item{\textbf{Architect/Technique}{: SOLID Design Principles, Event-driven, Software design pattern, Multithreaded Programming}}
    \vspace{-8pt}
    \item{\textbf{Deeplearning}{: Cuda, TensorRT, ONNX, OpenCL, PyTorch, OpenCV, PaddlePaddle, TensorFlow}}
    }
    
\resumeSubHeadingListEnd

%-----------EXPERIENCE-----------------
\shorterSection{\textcolor{titleblue}{Experience}}
  \resumeSubHeadingListStart

    \resumeSubheading
  {ThinkTron Ltd.}{Siyuan St, Zhongzheng District, Taipei City, Taiwan}
  {AI Software Engineer}{May 2025 - Present}
  \resumeItemListStart
    \resumeItem{Project: Satellite Imagery AI Automatic Identification System}
            {
            \begin{itemize}
                \item Oriented Object Detection: Applied \textbf{YOLOv11-OBB} to perform \textbf{Oriented bounding boxes Object Detection} for accurately capturing the orientation of aircraft, ships, and vehicles.
                \item Small Object Enhancement: Integrated \textbf{Slicing Aided Hyper Inference (SAHI)} to slice remote sensing images for parallel inference, significantly boosting the recall rate for small targets. Leveraged a Super-Resolution Model to enhance image detail, improving subsequent detection and annotation quality.
                \item GPU Performance Optimization: Utilized \textbf{TensorRT} for FP16 quantization and layer fusion to achieve stable, low-latency inference.
                \item AI Serving \& Deployment: Deployed models on NVIDIA Triton Inference Server, integrating with a FastAPI backend via gRPC/API. Applied \textbf{ Distributed Data Parallel (DDP)} across multiple GPUs to ensure efficient execution of large-scale inference tasks.
                \item Experiment Management: Used \textbf{MLflow} to track model versions, hyperparameters, and performance metrics
                \item Parallel Processing: Automated data workflows using \textbf{Apache Airflow}, employing dynamic task slicing to process large aerial images in parallel (Pansharpening, COG generation, PostGIS Raster writing, GeoServer publishing).
                \item Resource Management: Integrated GDAL to build partition indexes at each stage and used Airflow's resource-aware scheduling to optimize CPU I/O allocation.
            \end{itemize}
            }
    \resumeItemListEnd
    
\resumeSubheading
  {FUBON LIFE INSURANCE CO., LTD.}{Neihu District, Taipei City, Taiwan}
  {Data Scientist Intern}{Mar 2024 - Sep 2024}
  \resumeItemListStart
    \resumeItem{Project: Insurance AI}
            {
            \begin{itemize}
                \item Image Preprocessing: Implemented advanced image processing algorithms with \textbf{OpenCV} to enhance document clarity.Developed and applied a \textbf{Non-Maximum Suppression (NMS)} algorithm to specifically mitigate text occlusion from document stamps, significantly improving the accuracy of downstream text detection models. 
    \item \textbf{OCR Engine Development:}
    \begin{itemize}
        \item \textbf{Text Detection:} Deployed and fine-tuned models from the \textbf{ICDAR benchmark (DBNET, EAST)} for robust text localization. Systematically evaluated model performance against benchmarks using \textbf{Recall, Precision, and F1-score}.
        \item \textbf{Text Recognition:} Integrated \textbf{CRNN, SVTR, and TROCR} models for high-accuracy character transcription, leveraging open data from \textbf{ESUN.AI}. Validated model performance using \textbf{Character Error Rate (CER)}.
        \item \textbf{Data Augmentation:} Utilized \textbf{Text Renderer} (an OCR generator tool) to programmatically generate a large-scale synthetic training dataset, enhancing model generalization and robustness against varied document fonts and formats.
    \end{itemize}
                
                \item Designed microservices deployed on Kubernetes , Using Grafana for model performance metrics and Kibana for log analysis
            \end{itemize}
            \item \textbf{Natural Language Processing (NLP) \& Document Intelligence:}
    \begin{itemize}
        \item Enhanced the LLM's OCR capabilities by integrating \textbf{layout-aware embedding} techniques with \textbf{LayoutXLM}.
        \item Appliied TAIDE encoder model for Traditional Chinese text encoding, reducing character segmentation errors by 45\%,Optimized model performance through DPO and LoRA , achieving 3x faster inference while maintaining 92\% accuracy
        \item Fine-tuned a \textbf{Llama 2.0} model for semantic recognition and entity extraction using advanced techniques including \textbf{Direct Preference Optimization (DPO)} and \textbf{Low-Rank Adaptation (LoRA)} from the \textbf{PEFT} library.
        \item Optimized and deployed the fine-tuned model for efficient inference using \textbf{Llama.cpp}.
        
    \end{itemize}
    }
    \resumeItemListEnd

\resumeSubheading
{Knowledge Engineering Lab, Fu Jen Catholic University}{New Taipei city, Taiwan}
{Research Assistant, Supervisor: \textbf{Hsu Jia-Lien}}{Jul 2022 - Sep 2024}
\resumeItemListStart
\resumeItem{Form Understanding Project Certificate of Diagnosis }
{Architected end-to-end medical document analysis system on ALTOS AI workstation using Kubernetes orchestration.
Integrated OCR algorithms and document understanding models in containerized microservices architecture. Earned project scholarship award for innovative system design with FUBON LIFE INSURANCE
(\href{https://www.fju.edu.tw/focusDetail.jsp?focusID=2285&focusClassID=3&fbclid=IwAR1U61iocx-gGDaxGOqLT-qEkD5vceZsGNmUt_Yr5AQPWskePqsnKosxVh0}{\textcolor{linkblue}{link}})}
\resumeItemListEnd


\resumeSubheading
{Jasslin Technology Co., Ltd.}{Taishan District, New Taipei City, Taiwan}
{Part-time Software Engineer}{Jun 2019 - Sep 2022}
\resumeItemListStart
    \resumeItem{Project: Fleet Management System}{
      \begin{itemize}
          \item Leveraged Apache Kafka to develop a high-throughput event-driven platform for real-time data feeds of 10,000+ devices.
          \item Decoupled, scalable microservices for improved system flexibility and maintainability
          \item Utilized Java Socket.IO for efficient bi-directional communication between server and devices.
          \item Optimized database queries for handling large-scale inserts, achieving a 5x improvement in insert speeds, handling up to 10,000 inserts per second. Utilized SQL and NoSQL databases to efficiently handle different types of data
      \end{itemize}
    }
\resumeItem{Project: CRM System}{
    Enhanced user interfaces and workflows tailored to company needs.
}
\resumeItem{Project: Qualification License Management}{
    Developed a Quality License Management system with automated alerts for expiring licenses, integrated with the company's built-in CRM system.
}
\resumeItemListEnd

%-----------PROJECTS/SKILLS-----------------
\shorterSection{\textcolor{titleblue}{Projects}}
  \resumeSubHeadingListStart
    % \resumeSubItem{Visual SLAM}
    %  {
    %  Comprehensively \textbf{re-implemented} \href{https://github.com/rancheng/dso_understands}{\textcolor{linkblue}{DSO}} and annotated with exhaustive explains.
    %  }
    % \resumeSubItem{Deep Monocular Dense 3D Reconstruction}
    %     {Dense 3D reconstruction with \textbf{monodepth2} initialized Visual Odometry, leveraging traditional photometric consistancy, occlusion discrepancy, and local geometrical-smooth assumptions to \textbf{optimize depth estimation} (\textbf{LM} method) and \textbf{register} 3D map point clouds.
    %     }
    % \resumeSubItem{Abstraction Augmented Deep RL}
    %   {Abstract rgb image with Unet shaped network to digest image in latent representation, and learn from latent inputs, average convergence time increased 27.3\%, maximum reward (10M iterations) is 1.21 times than baseline model without abstraction augmentation, experiments conducted under self-collected dataset from AirSim simulator (\href{https://github.com/rancheng/AirSimProjects}{\textcolor{linkblue}{github}})
    %   }
    % \resumeSubItem{Forgetting Model for BP}
    %   {Introduced forgetting model for back propagation as in gradient dynamic routine, inspired from forgetting curve, I invented forgetting factor to regulate delta weights updates (\href{https://rancheng.github.io/forgetting-model/}{\textcolor{linkblue}{math proof}})}
    % \resumeSubItem{LOAM}{extended LOAM (LiDAR Odometry and Mapping) with co-visibility check, optimized with Ceres optimizer and asynchronous threading}
    \resumeSubItem{CB-STM Test Implementation}{ Developed an Android mobile application for early Mild Cognitive Impairment (MCI) detection. Engineered real-time user behavior tracking system using SQLite database and implemented ML models optimized with ONNX for on-device predictions. Successfully processed 1000+ assessments and demonstrated at Taiwan HealthCare Expo 2023  (\href{https://www.chinatimes.com/realtimenews/20221205001624-260421?chdtv}{\textcolor{linkblue}{link}})}
    \resumeSubHeadingListEnd
    \fill

% -----------Addtional Experience & Achievements-----------------
\shorterSection{\textcolor{titleblue}{Publications}}
    \resumeSubHeadingListStart
      \small
        \item {Adaptive Model Transfers and Aggregations for Efficient Federated Learning in IoT Edge Systems with Non-IID Data,
        Jenn-Wei Lin, Taewoon Kim, \textbf{Hung-Jen Tu},Po-Hsien Kuo,
        \textbf{DSIT 2023},
        \href{https://ieeexplore.ieee.org/document/10423980}{\textcolor{linkblue}{paper}}
    \resumeSubHeadingListEnd
\shorterSection{\textcolor{titleblue}{Certificates }}
  \resumeSubHeadingListStart
  \small
    \item { \href{https://camo.githubusercontent.com/6e76155581f50a3496bfdcab2008cdc3902839d189be6c39fc8412f69833d577/68747470733a2f2f6170692e61636372656469626c652e636f6d2f76312f66726f6e74656e642f63726564656e7469616c5f776562736974655f656d6265645f696d6167652f63657274696669636174652f3834313936363033}{\textcolor{linkblue}{TensorFlow Developer Certificate}
        , Google
\hfill
\vspace{-8pt}}
    \item { \href{https://www.coursera.org/account/accomplishments/specialization/TNVAQWY9TTTA?}{\textcolor{linkblue}{Deep Learning Specialization}},
        DeepLearning.AI
    \hfill
    \resumeSubHeadingListEnd
\shorterSection{\textcolor{titleblue}{Languages}}
\resumeSubHeadingListStart
\small{
    \item{\textbf{Vietnamese}{: Native}}
    \vspace{-8pt}
    \item{\textbf{Chinese}{: Proficient}}
    \vspace{-8pt}
    \item{\textbf{English}{: Proficient}}
}
\resumeSubHeadingListEnd
%-------------------------------------------
\end{document}
